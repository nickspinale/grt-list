\dfn{
    Let $G$ be a graph.
    To each vertex in $G$, assign an ordering of its neighbors.
    This order is called a preference list.
    A matching is stable if there is an edge not in the matching such that each endpoint either prefers the other over its match or is unmatched.
    Such an edge is called an unstabilizing edge.
}

\thm[Gale-Shapley]{
    A bipartite graph always has a stable matching.
}

\dfn{
    A proper coloring of a graph $G$ is a function $c : V(G) \rightarrow A$
    such that $\forall uv \in E(G) : c(u) \neq c(v)$.
}

\dfn{
    The chromatic number of a graph $G$, denoted $\chi(G)$, is the minimum $n \in \N$
    such that there exists some proper coloring $c : V(G) \rightarrow [n]$.
}

\fact{
    $\chi(G) \leq \Delta(G) + 1$.
}

\fact{
    $\chi(K_n) = n$
}

\fact{
    $\chi(C_{2k+1}) = 3$
}

\thm[Brooks]{
    If $G$ is connected and not isomorphic to the a complete graph or an odd cycle, then $\chi(G) \leq \Delta(G)$.
}

\prop{
    $\chi(G) \leq \text{ max } \{\;\delta(H) : \text{$H$ is an induced subgraph in $G$} \;\} + 1$.
}

\dfn{
    The clique number of $G$, denoted $\omega(G)$, is the size of the largest complete subgraph in $G$.
}

\prop{
    $\chi(G) \ge \omega(G)$
}

\homework

\prop{
    If a graph is not bipartite, then there exists an assignment of preference lists for which no stable matching exists.
}
